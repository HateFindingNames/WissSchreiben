\begin{abstract}
    In den medialen Landschaften ist Aufmerksamkeit als eine Ressource zu verstehen, um die Medienhäuser weltweit zu ringen versuchen.
    Als Thema globalen Ausmaßes platziert sich der menschengemachte Klimawandel zwischen kriegerischen Auseinandersetzungen, Flüchtlingsströmen, Wirtschaftskrisen und Populismus.
    Zwischen globalem Norden und globalem Süden existieren deutliche Unterschiede in der Art und Intensität mit der Menschen den Klimawandel wahrnehmen.
    Regulatorische Maßnahmen zur Reduktion des \(CO_2\)-Ausstoßes in reichen Ländern verändern traditionelle Lebensentwürfe und -weisen.
    Dem gegenüber stehen Extremwetterereignisse und dauerhafte klimatische Veränderungen im globalen Süden.
    Deutliche Unterschiede in inhaltlicher Tiefe und Frequenz der Berichterstattung lassen sich unter anderem inhaltlicher Tiefe, vor allem aber Frequenz der Nachrichtenbeiträge mit Bezug zu Klimawandel zu beobachten.
    Erklärungen zu den deutlichen unterschieden in inhaltlicher Tiefe und Frequenz der Berichterstattung liefert die starke Ungleichheit der Bildung in den Bevölkerungen, aus denen sich schließlich auch die Medienschaffenden rekrutieren. 
    \par\vskip\baselineskip\noindent
    \textbf{Keywords: Medien, Kommunikation, Klimawandel, Klima, Politik, Demografie, Bildung}
\end{abstract}