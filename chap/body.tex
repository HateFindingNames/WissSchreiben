\section{Einleitung}

Die menschengemachte Klimakatastrophe ist ein globales Phänomen, dem sich keine Menschengruppe vollständig entziehen kann~\cite{Book.Calvin.IPCCClimateChange2023SynthesisReport.2023}.
Der subjektiven Wahrnehmung nach ist es für Individuen jedoch durchaus möglich, die Augen davor zu verschließen.
Durch die zeitliche und geografische Entkopplung von klimaschädlichem Verhalten~--~Emission von klimaschädlichen Gasen~--~und seinen Auswirkungen findet die gesellschaftliche Perzeption durch überlieferte Erzählungen statt~\cite{Article.Moser.CommunicatingClimateChangeHistoryChallengesProcessAndFutureDirections.2009}. % satz überarbeiten
Die Wahrnehmung der Existenz klimatischer Veränderungen und deren Auswirkungen auf globaler und lokaler Ebene, sowie die Kontextuierung durch die wahrnehmenden Gesellschaften ist neben Faktoren wie allgemeiner und wissenschaftlicher Bildung von der Berichterstattung und der Reaktanz verschiedener Akteure innerhalb dieser Gesellschaften abhängig.
Hauptakteure sind neben Politik und den etablierten Medienhäusern vermehrt auch Personen und Institutionen auf Plattformen sozialer Medien wie X/Twitter, Instagram und TikTok.
Bisweilen nehmen auch Betreiber der Plattformen wie etwa Elon Musk oder Jeff Bezos am (globalen)Diskurs direkt durch eigene Aussagen und Meinungen oder indirekt durch entsprechend gestaltete Unternehmenspolitik Teil.\par\medskip

Ebenso, wie klimaschädliche Emissionen global sehr heterogen verteilt sind~--~die kumulative \(CO_2\)-Emission von China, USA und der EU verantworten mehr als \qty{50}{\percent} der Gesamtemissionen~\cite{Misc.Centre.GHGEmissionsOfAllWorldCountries.2023}~--~so sind es auch ihre Auswirkungen.\par
Die vorliegende Arbeit geht der Frage nach, ob und inwiefern eine Korrelation mit der medialen Berichterstattung respektive der politischen Kommunikation zu beobachten ist.
Außerdem wird untersucht, ob Unterschiede im gesellschaftlichen Diskurs insbesondere in den sozialen Medien zu beobachten sind.

\section{Globale Emissionsverteilung}

Bekannt und durch breite Studien gestützt sind die Industrienationen und die entwickelten Länder maßgeblich an den absoluten Emissionen von klimaschädlichen Gasen verantwortlich sind mit den Hauptemittenten China (), USA () und Indien ().% werte und quelle angeben
Die pro-Kopf Emissionen jedoch sind vergleichsweise homogen über den Globus verteilt.
\Cref{fig:ghg per kopf} zeigt eine farbcodierte Karte Emissionen in Tonnen pro Kopf für das Jahr 2022.
An der Spitze sind die Öl fördernden Länder des Mittleren Ostens, der Norden Asiens, die USA, Kanada und China.
Am unteren Ende befinden sich die Länder Zentralafrikas~\cite{Misc.DataPagePerCapitaCOEmissions.2023}.

{%
\vspace{\baselineskip}
\centering
\includesvg[width=\linewidth]{stat/co-emissions-per-capita}
\captionof{figure}{Weltweite Treibhausgasemissionen in Tonnen pro Kopf und Jahr für das Jahr 2022\cite{Misc.Centre.GHGEmissionsOfAllWorldCountries.2023}.}\label{fig:ghg per kopf}
\vspace{\baselineskip}
}

% Als globaler Süden werden hier Länder der sub-Sahara Zone, Südamerika ...

\section{Der globale Norden}

% ToDo:
% * russland-bezogene artikel suchen. kurze suche hat nicht viel ergeben..

\cite{Misc.ReitisMuenstermann.MonitoringClimateChangeAndCOP28InEuropeanOnlineNewsMedia.2024}

Die in \cref{fig:omm} gezeigte Karte kodiert farblich den relativen Anteil klimabezogener Nachrichtenbeiträge relativ zur jeweils nationalen Gesamtanzahl der Nachrichtenbeiträge zwischen dem 1. Januar 2017 und 1. September 2024\footnote{Der Datensatz wird täglich aktualisiert.}\footnote{Nähere Erläuterung des Datensatzes unter \url{https://icdc.cen.uni-hamburg.de/omm/OnlineMediaMonitor_TheGuide.pdf}.}.
Betrachtete Länder sind \glqq Argentinien, Australien, Brasilien, Deutschland, Großbritannien, Indien, Italien, Kanada, Malaysia, Neuseeland, Norwegen, Polen, Schweden, Schweiz, Singapur, Spanien, Südafrika und die USA\grqq.
Auffällig ist sofort die gegenüber allen übrigen Nationen höhere Aktivität in Nordamerika nämlich Kanada und den USA.\par

{%
\vspace{\baselineskip}
\centering
\includesvg[width=\linewidth]{stat/world_climate_news_coverage/images/fig1}
\captionof{figure}{Kumulative Häufigkeitsverteilung der Artikel mit Klimabezug in Printmedien von 2004 bis 2024~\cites{Dataset.Boykoff.EuropeanNewspaperCoverageOfClimateChangeOrGlobalWarming20042024July2024.2024}{Dataset.Boykoff.MiddleEasternNewspaperCoverageOfClimateChangeOrGlobalWarming20042024July2024.2024}{Dataset.Boykoff.NorthAmericanNewspaperCoverageOfClimateChangeOrGlobalWarming20002024July2024.2024}{Dataset.JimenezGomez.LatinAmericanNewspaperCoverageOfClimateChangeOrGlobalWarming20052024July2024.2024}{Dataset.NacuSchmidt.AfricanNewspaperCoverageOfClimateChangeOrGlobalWarming20042024July2024.2024}{Dataset.Oonk.OceaniaNewspaperCoverageOfClimateChangeOrGlobalWarming20002024July2024.2024}{Dataset.Pearman.AsianNewspaperCoverageOfClimateChangeOrGlobalWarming20042024July2024.2024}.}\label{fig:omm}
\vspace{\baselineskip}
}

Nachrichtenhäuser und -institutionen sind in den kapitalistischen Mechanismus der Industrialisierung des Medienkonsums oder die Kulturindustrie eingebettet~\cite{Book.Horkheimer.Kulturindustrie.2024}; die Auswahl erwähnenswerter Nachrichten erfolgt nicht nur etwa nach objektiver Relevanz~--~wie viele sind wie schwer betroffen~--~sondern nach zu generierenden Auflagen, Klicks oder Einschaltquoten.
\glqq Der Mensch als mit Welt beliefertes Wesen\grqq lenkt mit seiner Aufmerksamkeit den thematischen Fokus der ihm von den Medienschaffenden gelieferten Inhalte~\cite{InCollection.Anders.DieWeltAlsPhantomUndMatrize.2019}.
Solche Inhalte, die in der Bevölkerung wenig Aufmerksamkeit zu binden vermögen, unterliegen und werden folglich unterrepräsentiert.


\section{Der globale Süden}

Aufgrund heterogenerer kultureller Hintergründe Medienschaffender sowohl der Nord-Süd-Achse, als auch innerhalb von Ländern des globalen Südens~--~etwa Südamerikas und Zentralafrikas~--~ist es erforderlich, Unterschiede in der Berichterstattung innerhalb ihrer jeweiligen Kontexte zu betrachten.

Jährliche Berichte des UNICEF attestieren Afrika noch immer die problematischste Bildungssituation weltweit.
Bildung unterstützende Technologien werden zwar stetig günstiger,
dem \textit{Global Education Monitoring} Bericht der UNESCO von 2023 nach verfügen große Teile der afrikanischen Bevölkerung dennoch über keinen Internetzugang~\cite{Book.2023GEMReportSummaryTechnologyInEducationAToolOnWhoseTerms.2023}.
Mit Ausnahme von Südafrika, São Tomé/Príncipe und Zimbabwe beträgt die Zahl der Personen mit formaler Primärbildung flächendeckend weniger als \qty{90}{\percent}~\cite{Article..GlobalEducationMonitoringReport.}.
Einen höheren Sekundarabschluss erreichen in allen betrachteten Ländern deutlich unter \qty{75}{\percent}.\par\medskip

So verfügen die Rezipienten der Kommunikation des menschengemachten Klimawandels und seinen Auswirkungen selten über die notwendige Vorbildung um etwa komplexe atmosphärische Zusammenhänge verstehen zu können.
Obschon Umfragen zufolge sich afrikanische Journalist:innen der Klimakatastrophe und ihrer Auswirkungen sehr bewusst sind, begrenzen sich Nachrichtenbeiträge häufig auf lokale Extremereignisse~\cite{TechReport.Lidubwi.ClimateJournalisminEastAfricaInAnEraOfMisinformation.2023}. % seite 7
Ursachenketten oder sich langsam entwickelnde, negative Veränderungen mit ihren Implikationen sind selten Inhalt~\cite{Booklet.Guedegbe.MEDIAPERCEPTIONSOFCLIMATECHANGEINSUBSAHARANAFRICA.2023}.% quelle
Einerseits liegt das begründet im allgemein geringen Bildungsstand bezüglich klimabezogener Zusammenhänge.
Als Primäreffekt ist andererseits jedoch auch hier~--~nicht anders als in etablierten Industrienationen~--~die im Mechanismus des Kapitalismus verortete Medien- und Nachrichtenindustrie:
finanziert werden solche Resorts, die in der Lage sind Aufmerksamkeit ihrer Audienz zu binden.
Bevölkerungsteile, die stark von ihrer unmittelbaren Umwelt abhängig sind, konsumieren eher Inhalte, die zeitlich und örtlich begrenzte Ereignisse thematisieren.
Doppelt-intensiviert tritt dieser Effekt durch die starke Korrelation zwischen Ländlichkeit und Abstinenz von formaler Bildung~\cite{Article..GlobalEducationMonitoringReport.}.\par\medskip

In Südamerika und den Ländern des Indopazifik ist der menschengemachte Klimawandel medial prävalenter.
Das mag mit einer insgesamt besseren Bildungssituation zusammen hängen~--~nahezu jedes Kind kommt zumindest zu einem primären Schulabschluss und höhere Sekundärabschlüsse sind deutlich weiter verbreitet.


\section{Private Akteure}